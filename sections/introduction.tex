\section{Objective}
The main objective is to make a streamlined prototype tool that can be used to analyze and verify RTIC applications. The RTIC framework abstracts a lot of details from the user. When the program is compiled it generates new code from the source code of the user. As KLEE works with symbolic values to generate its test cases, the input of the tasks in the generated code needs to be made symbolic. This step should result in new code generation with symbolic values which will be referred to as the test harness. KLEE can then generate test cases for each task from the test harness. Finally each test case can be replayed on the intended hardware or on emulated hardware. When replaying, the response time can measured and then be compared to a theoretical value. Currently there are no such tools available for this framework.

The tool should be able to do the following:
\begin{itemize}
    \item Generate code from RTIC that can be run symbolically (test harness)
    \item Generate KLEE tests on the generated code
    \item Replay test cases on actual or virtual hardware to measure execution time
    \item Detailed schedulability analysis based on the measurements
\end{itemize}
There will be some limitation to this tool to fit the time frame of the course. A minimal subset of the features of RTIC will be supported at a start, to try to prove that a tool like this can work. If there is time left, further features should be looked into.


