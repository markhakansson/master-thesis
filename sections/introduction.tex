For real-time computing systems, the processes running on them are liable to
meet their deadlines. I.e.\ the processes are bound to finish their execution
or respond within a certain time frame. This might be due to hardware
constraints; the system will not be able to work correctly if the system is
over-utilized. Or it could be critical that certain processes generate a
response in time, e.g.\ a sensor reading or an actuator.

\emph{Scheduling} is the act of assigning resources to the processes in a
system in an orderly fashion. Many algorithms and policies to schedule
processes have been introduced and mathematically proven to be correct. E.g.\
rate-monotonic scheduling, earliest deadline first scheduling and stack
resource policy amongst others [source here]. The act of verifying that the
system's processes adhere to the scheduler systems is called 
\emph{schedulability analysis}.

One way to do this analysis is to calculate the response time of each processes
and compare it to their respective deadlines. Known as \emph{response time analysis}.
If the response time is calculated to be larger than the deadline then it is said
to be unschedulable. In order to calculate the response time the worst-case
execution time (WCET) for the processes needs to be known. 

In this thesis the tool \emph{RAUK} for automatically verifying an embedded 
application's schedulability using a measurement based WCET analysis is presented. 
It currently only works on Run-Time Interrupt driven Concurrency (RTIC) applications,
which is a framework for programming real-time systems. Throughout this thesis 
\emph{process} and \emph{task} will be used interchangeably.


RAUK works in three steps. First it generates a test harness from the application
for which the symbolic execution tool \emph{KLEE} can generate test vectors for. 
Then it generates a replay harness from the application which is used to replay
the generated test vectors on actual hardware. Finally the WCET analysis is
measured using those test vectors running on the hardware itself.

\section{Objective}
The main objective is to make a streamlined prototype tool that can be used to
analyze and verify RTIC applications. The RTIC framework abstracts a lot of
details from the user. When the program is compiled it generates new code from
the source code of the user. As KLEE works with symbolic values to generate its
test cases, the input of the tasks in the generated code needs to be made
symbolic. This step should result in new code generation with symbolic values
which will be referred to as the test harness. KLEE can then generate test
cases for each task from the test harness. Finally each test case can be
replayed on the intended hardware or on emulated hardware. When replaying, the
response time can measured and then be compared to a theoretical value.
Currently there are no such tools available for this framework.

The tool should be able to do the following:
%
\begin{enumerate}
    \item Generate code from RTIC that can be run symbolically (test harness)
    \item Generate KLEE tests on the generated code
    \item Replay test cases on actual or virtual hardware to measure execution time
    \item Detailed schedulability analysis based on the measurements
\end{enumerate}

\section{Limitiations}
There will be some limitation to this tool to fit the time frame of the course.
A minimal subset of the features of RTIC will be supported at a start, to try
to prove that a tool like this can work. If there is time left, further
features should be looked into. 

\section{Previous Work}
Worst case execution time (WCET) analysis with KLEE has been tried in an earlier
version of the RTIC framework\cite{lindner}. Although a lot of it is now
outdated and the script was not very streamlined. The analysis script also 
utilized a lot of tools not made in the Rust language, which does not integrate
very easy into the Rust ecosystem.

