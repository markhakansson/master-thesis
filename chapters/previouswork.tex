\section{Measurement-based WCET analysis}
A measurement-based worst-case execution time (WCET) analysis with KLEE has
been tried in an earlier revision of the RTIC framework when it was called
RTFM\cite{lindner}. In the paper the authors presented their approach for using
KLEE on RTFM applications in order to measure the WCET for all user tasks
using the GNU debugger (GDB) which was automated by Python scripting.

In their approach they modified the RTFM application by forking the library and
adding two additional compilation flags. One which modified the application to be
able to execute KLEE on the application to generate test vectors, and another
to modify the application to make it possible to replay said test vectors on
the target hardware.

KLEE was used to generate test vectors for all resources that the user tasks
had access to. On the target hardware the modified RTFM application would be
running with software breakpoints inserted at the start and end of all tasks
and resource locks. By using GDB they would for each test vector on a single
task, write the corresponding test vector values to the memory locations of the
resources. Then execute the program and step through the breakpoints until the
task has finished its execution. On each breakpoint they would record the
current cycle count. By doing this they would get the execution times of all
tasks and their resource claims. Since the KLEE interpreter executes all
feasible paths in the program, one of the test vectors should in theory yield
the WCET of a task.

Some limitations of their approach include
\begin{enumerate}
    \item it only worked on simple resource types,
    \item it did not model I/O.
\end{enumerate}
In the paper they only showed their work on an RTFM application where all
resources were integers and they mentioned that it did not model I/O. I.e.\
it did not create tests or measure resources which were connected to any I/O
such as hardware peripherals, serial ports etc. Although they mentioned
schedulability analysis in the paper, they did not showcase any such examples
where their results were used to run a schedulability analysis check.

The paper by Lindner et al.\ has served as the main inspiration to this thesis.
Thus this thesis can be seen as the natural evolution of their work.

\section{Schedulability/WCRT analysis}
A few related tools that is currently freely available for running a
schedulability analysis on real-time systems are Cheddar\cite{cheddar},
TIMES\cite{timestool} and RT-Druid\cite{rtdruid}. In all the tools the
real-time system is modelled by the user, provided with the tasks in the system
and their behaviours. The modelled system can then be simulated and verified
for its schedulability using the scheduling policy of the system. Both RT-Druid
and the Cheddar tool are still actively maintained and the latter is used in
some commercial applications\cite{ellidiss}.

The downside of these tools is that the system needs to be modelled by hand
(in Cheddar's case, using an architecture design language). It will require
developers to maintain both the real-time system itself and a model of it.
Whilst necessary for safety critical systems, the need to model the real-time
system could be seen as an obstacle for many smaller projects and the
open-source community in general to test and verify their systems for
schedulability.

In the commercial sector RapidRMA\cite{rapidrma} provides a complete toolset
for integrating schedulability analysis in embedded applications. It is not
openly available and difficult to compare to the other tools.

Whilst researching the related work, no currently available schedulability
analysis tools were found for the RTIC framework.
