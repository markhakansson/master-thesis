\section{Measurement-based WCET analysis}
A measurement-based worst-case execution time (WCET) analysis with KLEE has
been tried in an earlier revision of the RTIC framework\cite{lindner} when it
was called RTFM. In the paper the authors presented their approach for using
KLEE on RTFM applications in order to measure the WCET for all user tasks as a
Python script.

In their approach they modified the RTFM application by forking the library and
adding two additional compilation flags. One which modified the application to be
able to execute KLEE on the application to generate test vectors, and another
to modify the application to make it possible to replay said test vectors on
the target hardware.

KLEE was used to generate test vectors for all resources that the user tasks
had access to. On the target hardware the modified RTFM application would be
running with software breakpoints inserted at the start and end of all tasks
and resource locks. By using the \texttt{gdb} debugger they would for each test
vector on a single task, write the corresponding test vector values to the
memory locations of the resources. Then execute the program and step through the
breakpoints until the task has finished its execution. On each breakpoint
they would record the cycle counter. By doing this they would get the execution
times of all tasks and their resource claims. Since the KLEE interpreter
executes all feasible paths in the program, one of the test vectors should
in theory yield the WCET of a task.


Some limitations of their approach include
\begin{enumerate}
    \item it only worked on simple resource types,
    \item it did not model I/O.
\end{enumerate}
In the paper they only showed their work on an RTFM application where all
resources were integers and they mentioned that it did not model I/O. I.e.\
it did not create tests or measure resources which were connected to any I/O
such as access to hardware peripherals. Although they mentioned schedulability
analysis in the paper, they did not showcase any such examples where their
results were used to run a schedulability analysis.

The paper by Lindner et al.\ has served as the main inspiration to this thesis.
Thus this thesis can be seen as an evolution of their work.

\section{Schedulability analysis}
Cheddar, TIMES 
