\section{Rust programming language}

\section{Stack Resource Policy}
Stack Resource Policy\cite{srp} is a policy that defines how shared resources
should be accessed in a system, to prevent deadlocks and bounded priority
inversion in an efficient manner. It works with schedulers with fixed or
dynamic priorities on tasks such as rate monotonic (RM) or earliest-deadline
first (EDF). 

\section{RTIC}
Real-Time Interrupt-driven Concurrency or RTIC for short, is a concurrency
framework for embedded systems based on the Stack Resource Policy for
scheduling tasks. It based on the RTFM framework\cite{rtfm}. It abstracts many
details of the underlying model from its users without any significant overhead
and a guaranteed deadlock-free execution. In RTIC the user defines the
application by defining the shared resources, the initialization function and
the \emph{tasks} in the system. The initialization function is the first
function to run in the application. It has complete access to the hardware and
is used to initialize the application and the hardware itself, as well as
initializing any shared resources that can only be set during run-time.

A \emph{task} is the unit of concurrency used within RTIC. It can be bound to
an interrupt vector, then it becomes a \emph{hardware task}. Which will be
triggered on a certain hardware event. Or not, then it is a \emph{software
task}. Which can be triggered inside the application itself. All tasks can be
prioritized and the priorities will be fixed.

An example of shared resources in an RTIC application can be seen in Listing
\ref{lst:shareresources}. Here the \texttt{integer} is set to be initialized to
$0$ at compile-time whereas the \texttt{input} input pin will have to be initialized
correctly at run-time.
\lstinputlisting[
    language={rust},
    label={lst:shareresources},
    float=h,
    caption={RTIC shared resources}
]{../code/rtic_resources.rs}

A task accessing the shared resources can be seen in Listing \ref{lst:task}.
All shared resources that tasks needs access to will have to be declared in the
task attributes. To get access to the shared resource \texttt{input} inside the
task it needs to be locked. Once a resource is locked it will have exclusive
access to it.
\lstinputlisting[
    language={rust},
    label={lst:task},
    float=h,
    caption="RTIC task"
]{../code/rtic_task.rs}

When the user compiles their RTIC application the application will be expanded
with all necessary details. Using Rust's procedural macros. It will take the
information in the attributes and create helper functions for all tasks and
resources and connect tasks with dispatchers. As well as functions for updating
resource priorities which is required to adhere to the stack resource policy.

RTIC has very little overhead on the application. This is because it lets the
hardware itself handle most of the scheduling work using ARM's nested vector
interrupt controller (NVIC).  The overhead RTIC has mostly concerns context switching
as well as task dispatching.

\section{Schedulability Analysis}
Response Time Analysis from \textit{Hard Real-Time Computing
Systems}\cite{hardrealtimecomputingsystems}
\begin{equation}
    \begin{cases}
        R_{i}^{(0)} &= C_i + B_i \\
        R_{i}^{(s)} &= C_i + B_i + \sum\limits_{h: P_h > P_i} \left\lceil \frac{R_{i}^{s-1}}{T_h} \right\rceil
    \end{cases}
\end{equation}


\section{KLEE}
KLEE\cite{kleepaper} is a tool to generate test vectors for programs with the
ability to replay them, using symbolic execution. Compared to fuzz testing
where program input usually consists of randomized concrete values, symbolic
execution handles input as symbolic values which are arbitrary. These values
are handled by an interpreter which executes the program and keeps track of the
states where the symbolic values are used for branching paths (e.g.\
conditional branches such as `if' and `else') and modifications to and/or using
those values. Then for each state it utilizes an SMT solver to find for which
values the symbolic values should be in order to enter those branches or
invalid modifications. This could be the case where an array is accessed using
a symbolic value as the index. The SMT solver will find that the array will be
accessed out of bonds for certain values of it. Thus producing two branches,
one where the array is not accessed out of bonds and one where it is. As the
symbolic value changes over the program, the intepreter creates a tree for all
possible branching paths of its value. Finally when the program terminates KLEE
will output the test vectors and the user can replay those test vectors on the
program.
