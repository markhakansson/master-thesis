In this thesis the tool RAUK for automatic schedulability analysis of embedded
RTIC applications using symbolic execution is presented. RAUK can on simple
RTIC applications determine whether the tasks running in the system are
schedulable or not with very little changes to the application by the user.

The motivation for this tool is to improve the current approaches for
schedulability analysis in general and improve the tooling for embedded
systems in the Rust ecosystem. Current approaches to schedulability
analysis requires a separate model to be defined. Which then requires
the developers to keep both their application and the model up-to-date.
Currently there are no schedulability analysis tools for any embedded Rust
systems. Safety-critical systems for certain industries e.g. automotive and
aviation industry needs to adhere to timing requirements amongst others. To
pave a way for RTIC to be used for such systems, tools needs to available to
determine that they too, can adhere to the same requirements.

Using the symbolic execution tool KLEE, test vectors are generated targeting
all execution paths in all user tasks in the RTIC application. The test vectors
are replayed and their execution times are recorded on the target hardware.
Since the test vectors target all execution paths, the measurements will
include the worst-case execution time (WCET) for all tasks. These WCET
measurements are used to determine whether the system is schedulable or not.

The current implementation of RAUK has some constant overhead in the
measurements. When compared to tasks which generally have a long execution
time, this overhead is comparably small. But when considering tasks
with low execution times this overhead could lead to the measurement
being several times larger than the actual execution time.
