In this thesis the proof-of-concept tool RAUK for automatic schedulability
analysis of embedded RTIC applications using symbolic execution is presented.
The RTIC framework provides a declarative executable model for building
embedded applications, which behaviour is based on established formal methods
and policies, thus is amenable for both WCET and scheduling analysis techniques.

Utilizing the symbolic execution tool KLEE, test vectors are generated covering all
feasible execution paths in all user tasks in the RTIC application.
Additionally KLEE will generate test vector for any possible errors e.g.
arithmetic or array indexing errors found. The test vectors are replayed on the
target hardware to record a worst-case execution time (WCET) estimation for all
tasks. These WCET measurements are used to determine if the system is
schedulable using formal scheduling analysis techniques.

The work in this thesis is based on previous research in this field for WCET
estimation using KLEE on an older iteration of the RTIC framework. Our
contributions include a focus on the latest RTIC version and a seamless
integration with the Rust ecosystem as well an automatic scheduling analysis.

The evaluation of this tool shows that RAUK can add some substantial overhead
to the WCET estimations and that certain types of paths can be missing from the
built binaries used for the analysis of the application.
