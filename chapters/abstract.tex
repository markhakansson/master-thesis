In this thesis, the proof-of-concept tool RAUK for automatically analyzing
RTIC applications for schedulability using symbolic execution is presented.
The RTIC framework provides a declarative executable model for building
embedded applications, which behavior is based on established formal methods
and policies. Because of this, RTIC applications are amenable for both
worst-case execution time (WCET) and scheduling analysis techniques.

Internally, RAUK utilizes the symbolic execution tool KLEE to generate test vectors
covering all feasible execution paths in all user tasks in the RTIC
application. Since KLEE also checks for possible program errors e.g.\ arithmetic or
array indexing errors, it can be used via RAUK to verify the robustness of the
application in terms of program errors. The test vectors are replayed on the
target hardware in order to record a WCET estimation for all user
tasks. These WCET measurements are used to derive a worst-case response time
for each user task, which in turn is used to determine if the system is
schedulable using formal scheduling analysis techniques.

% improve evaluation
The evaluation of this tool shows a good correlation between the results from
RAUK and manual measurements of the same tasks, which showcases the viability
of this approach. However, the current implementation can add some substantial
overhead to the measurements and sometimes certain types of paths in the
application can be completely absent from the analysis.

The work in this thesis is based on previous research in this field for WCET
estimation using KLEE on an older iteration of the RTIC framework. Our
contributions include a focus on an RTIC 1.0 pre-release, a seamless
integration with the Rust ecosystem, minimal changes required to the
application itself, as well as an included automatic schedulability analyzer.

% add future (i.e. good groundwork for improvements)
Currently, RAUK can verify simple RTIC applications for both program errors and
schedulability with minimal changes to the application source code. The
groundwork is laid out for further improvements that are required to function
on larger and more complex applications. Solutions for known problems, and future work
are discussed in Chapters~\ref{chapter:discussion},~\ref{chapter:conclusion} respectively.
