The goal of this thesis was to explore the possibilities and limitations of
developing a tool to automatically calculate the WRCT for user tasks in an
RTIC application. 

Inspired by the previous work by \cite{lindner} the thesis resulted in the tool
RAUK for response-time analysis of RTIC applications using the symbolic
execution tool KLEE for automatic WCET measurements. With minimal changes to
the RTIC applications RAUK uses KLEE to generate test vectors in order to
target all paths in the tasks. Then the execution times for each test vector
will be measured on the target hardware. Using these measurements a
schedulability analysis can be performed with the SRP as basis, to check
whether the system is schedulable or not.

\section{Future work}
The problems presented in Chapter \ref{chapter:discussion}

This problem can be easily fixed by introducing side effects in the
aforementioned functions by e.g.\ do a volatile read to a memory address
inside them. By reading by writing directly to memory addresses, the Rust
compiler will not optimize out that specific code sections when the application
is built with optimizations enabled. This is because the compiler is unable to
determine if volatile memory accesses have side effects or not.

To combat this, the symbolic execution tool used in
RAUK would have to run on the hardware instructions instead of on LLVM IR. KLEE
does not support this however and only runs on top of LLVM.

Although there is a fix for this problem on the stable release channel in the
Cortex-M library by adding some additional flags to compiler, it was not
tested due to time constraints.

Unfortunately the schedulability analysis was only executed with RAUK and no
other available tools were tested. A comparison with other tools would have
been valuable to determine whether the results from the different tools woud
differ or not. If there would have been more time to allocate, this is one of
the key points that would have been most interesting to research further into.

To make this tool viable for continued support in the future, the tool would
benefit if many of the libraries in the RTIC and Cortex-M ecosystem for Rust
were more modularized. Many of the forked libraries are of larger libraries and
only changes small parts of them. If only smaller parts could be forked it
would significantly simplify the process of maintaining them. There currently
is work being made on modularizing the RTIC library itself, and then the
harnesses needed by RAUK could be easily extended in the RTIC library
without the need of maintaining the rest of the components.
