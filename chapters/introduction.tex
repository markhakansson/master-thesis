In computing a \emph{task} or sometimes \emph{process}, is a unit of work that
is executed or will be executed by the processor in a computer or a
microcontroller.  For many real-time computing systems, the tasks running on
them are liable to meet specific deadlines, i.e.,\ the tasks are bound to finish
their execution or respond within a certain time frame. This is especially true
for smaller embedded systems. This might be due to hardware constraints: the
system will not be able to work correctly if the system is over-utilized, or it
could be critical that certain tasks generate a response in time, e.g.,\ a
sensor reading or an actuator update. In such scenarios, a task that fails to
meet its deadline could lead to systematic failures. Systems where deadline
misses can cause systematic failures are called \emph{hard} real-time systems.
In comparison to \emph{soft} real-time systems where a deadline miss can
degrade the quality of the results produced by the system, but can be
tolerated~\cite{hardrealtimecomputingsystems}.

To make sure tasks adheres to their deadlines, many real-time systems utilize
a \emph{scheduler} that schedules all tasks. \emph{Scheduling} is the act of
assigning system resources, such as processor time or hardware peripheral
access to the tasks in an orderly fashion, such that all tasks can finish their
execution in rule with the scheduler's and tasks' constraints.

Many algorithms and policies to schedule tasks currently exists and have been
mathematically proven to be correct.  E.g.,\ rate-monotonic
scheduling~\cite{ratemonotonic} and earliest deadline first scheduling~\cite{edf}
amongst others.  The act of verifying that the system's tasks adhere to the
scheduler policy and are meeting their deadlines is called \emph{schedulability
analysis} or alternatively \emph{scheduling analysis}.

One way to execute this analysis is to calculate the longest time it takes for
each task to respond, i.e.,\ the longest time from when the task is ready to
execute until the end of its execution and compare it to their respective
deadlines.  In order to calculate the worst-case response time (WCRT), the
worst-case execution time (WCET) for the tasks need to be known. The
difference between WCRT and WCET is that WCET considers the longest execution
time of a task without any interruptions, whereas WCRT also considers
interruptions by tasks which have higher priority to
run and potential blocking from lower priority
tasks~\cite{hardrealtimecomputingsystems}. If the response time for any task is
calculated to be longer than its deadline, or the system is over-utilized then
the whole system is \emph{unschedulable}.

In this thesis, the tool \emph{RAUK} for automatically verifying an embedded
application's schedulability using a measurement-based WCET approach to
calculate the WCRT of all tasks is presented. Inspired by the previous work by
Lindner et al.\ \cite{lindner}. It currently runs on Run-Time Interrupt driven
Concurrency (RTIC) applications, which is a framework for programming real-time
systems for embedded devices in the Rust programming language. It currently
supports all ARM Cortex-M processors.

\section{Background}
Current approaches for automatic WCET calculation on real-time systems usually
utilizes two different methods combined:
\begin{enumerate}
    \item A measurement-based approach, analyzing smaller code sections running
        on the target hardware.
    \item Static analysis of the program using a model of the system.
\end{enumerate}
This is frequently referred to as the hybrid approach and is used in safety-
critical applications for the aviation industry~\cite{rapita} amongst others.
Previous research for the RTFM framework (now called RTIC) showed that it was
possible to utilize the symbolic execution engine KLEE~\cite{kleepaper} for a
measurement-based WCET analysis for all tasks~\cite{lindner}. It showcased the
possibilities, but their implementation was not very suitable for practical
applications as it could not analyze or measure WCET on I/O (Input/Output).

When it comes to schedulability analysis tools, there are very few active openly
available projects. Most such tools need the user to manually model the system
in order to verify it, as in the case of Cheddar~\cite{cheddar}. There have not
been any known previous attempts to create schedulability analysis tools for
the RTIC framework.

\section{Motivation}
For safety-critical systems developed for industries such as the automotive and
aviation industry, there is a recommendation and sometimes a requirement to
show that the systems strictly adhere to certain timing requirements. There
are tools for C-based languages such as RVS~\cite{rapita} that can analyze
execution times of the program and verify it against given requirements, but
there are no currently available tools for the RTIC framework or embedded Rust
that does something similar. In order to carve a path for RTIC and embedded
Rust in general to be used in such systems and industries, the tooling support
for program verification and analysis needs to improve drastically for Rust
programs to verify that RTIC applications can also meet the same requirements.

Additionally, the lack of easy-to-use tools can be seen as an obstacle for the
open-source community to test their applications. Most tools for schedulability
analysis are framework-agnostic, but because of that it requires a complete
model of the real-time system which are usually defined in obscure modeling
languages. A schedulability tool made specifically for RTIC with both
practicality and easy-of-use in mind, without the need to create a separate
model of the system for testing, could lead to more community-driven projects
to be tested more thoroughly.

\section{Problem definition}
This thesis explores the possibility of developing a tool to automatically
calculate the worst-case response times (WCRT) for tasks in an RTIC application
in order to check if the system is schedulable or not. This by utilizing the
symbolic execution engine KLEE to generate test vectors to target all execution
paths in the application. The assumption is that by targeting all execution
paths in a task, one of the paths will lead to the longest execution. Thus,
with the test vectors the WCET of all tasks should be possible to measure.

The thesis also explores what the limitations of such an approach would be, if
any. The goal is to ultimately apply the tool practically on actual working
RTIC applications.

\section{Delimitations}
The following are the high-level delimitations made in this thesis:
\begin{itemize}
    \item Software version delimitations.
    \begin{itemize}
        \item RTIC 1.0 pre-release
        \item Rust compiler version 1.51 (stable release channel)
    \end{itemize}
    \item No complex shared resource type support.
    \item Limited software task support.
    \begin{itemize}
        \item No Spawn API\@.
        \item No Schedule API\@.
    \end{itemize}
\end{itemize}
The RTIC framework has many features, and it would not be possible to test and
analyze all those within the time frame of this thesis. Because of this, the
focus has been on verifying tasks which accesses simple resources of primitive
types only and tasks that accesses hardware peripherals.

Complex resource types (e.g.,\ data structures) are much more difficult to work
with compared to primitive types (e.g.,\ integers, string slices) and are
unsupported due to it not being a high priority in order to create a
schedulability analysis tool.

In the RTIC framework, software tasks implement both a spawn and a schedule
API\@. This makes it possible to use a monotonic timer to easier execute software
tasks in the future after a set amount of time. These are unsupported as much
time would be needed to be spent on not only understanding the intrinsics of
the APIs but also in order to verify them correctly with this tool.

This thesis targets a pre-release of RTIC version 1.0 and is delimited to it.
Thus, it is not guaranteed that the work in this thesis will work on RTIC 1.0.

% \section{Thesis structure}
% This thesis starts with the introduction to the problem that RAUK solves and
% the background for it. Then some brief introduction to the related work that
% has been made in the field of WCET calculation and schedulability analysis.
% Particularly related to embedded Rust. In the theory chapter, the related
% theory behind the tools and techniques used in this thesis will be presented.
% Then follows the details of the implementation of RAUK and its main components.
% After that the evaluation of the implementation will be presented where the
% tool is tested on an RTIC application. Later follows the discussion of the
% results in the evaluation and its problems. The thesis ends with the conclusion
% and some notes about the future work of this tool.
