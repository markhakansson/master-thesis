In computing a \emph{task} or sometimes process, is a unit of work that is
executed or will be executed by the processor in a computer or microcontroller.
For many real-time computing systems, the tasks running on them are liable to
meet specific deadlines. I.e.\ the tasks are bound to finish their execution or
respond within a certain time frame. This is especially true for smaller
embedded systems. This might be due to hardware constraints; the system will
not be able to work correctly if the system is over-utilized. Or it could be
critical that certain tasks generate a response in time, e.g.\ a sensor reading
or an actuator. In such scenarios, a task that fails to meet its deadline could
lead to system failures.

To make sure tasks adheres to their deadlines, many real-time systems utilizes
a \emph{scheduler} that schedules all tasks. \emph{Scheduling} is the act of
assigning system resources, such as processor time or hardware peripheral
access to the tasks in an orderly fashion. Many algorithms and policies to
schedule tasks currently exists and have been mathematically proven to be
correct.  E.g.\ rate-monotonic scheduling\cite{ratemonotonic}, earliest
deadline first scheduling\cite{edf} and stack resource policy\cite{srp} amongst
others.  The act of verifying that the system's tasks adhere to the scheduler
policy and are meeting their deadlines is called \emph{schedulability
analysis}.

One way to do this analysis is to calculate the longest time it takes for each
task to respond, i.e.\ the longest time from when the task is ready to execute
until the end of its execution and compare it to their respective deadlines.
Known as \emph{response time analysis}. In order to calculate the worst-case
response time (WCRT) the worst-case execution time (WCET) for the processes
needs to be known. The difference between WCRT and WCET is that WCET considers
the longest execution time of a task without any interruptions whereas WCRT
also considers interruptions by tasks which have higher priority to
run\cite{hardrealtimecomputingsystems}. If the response time is calculated to
be larger than the deadline then it is said to be \emph{unschedulable}.

In this thesis, the tool \emph{RAUK} for automatically verifying an embedded
application's schedulability using a measurement based WCET approach to
caclulate the WCRT of all tasks is presented. It currently works on
Run-Time Interrupt driven Concurrency (RTIC) applications, which is a framework
for programming real-time systems for embedded devices in the Rust programming
language.

\section{Background}
% mention current wcet approaches
Current approaches for automatic WCET calculation on real-time systems usually
utilizes two different methods.
\begin{enumerate}
    \item Measurement-based approach analyzing smaller code sections on hardware
    \item Static analysis of the program using a model of the system
\end{enumerate}
This is frequently referred as the hybrid approach and is used in safety
critical applications for the aviation industry\cite{rapita} amongst others.
Previous research for the RTFM framework (now called RTIC) showed that it was
possible to utilize the symbolic execution engine KLEE for a measurement-based
WCET analysis for all tasks\cite{lindner}. It showcased the possiblities but
did not work for practical applications as it could not analyze or measure WCET
on I/O.

When it comes to schedulability analysis tools there are very few
active projects. Most such tools needs the user to manually model the system in
order to verify it like Cheddar\cite{cheddar}. There have not been any known
attempts for schedulability analysis tools for RTIC.

%RAUK works in three steps. First it generates a test harness from the application
%for which the symbolic execution tool \emph{KLEE} can generate test vectors for.
%Then it generates a replay harness from the application which is used to replay
%the generated test vectors on actual hardware. Finally the WCET analysis is
%measured using those test vectors running on the hardware itself.


\section{Motivation}
For safety-critical systems developed for industries such as automation and
aviation, there is a requirement to show that the system strictly adheres to
certain timing requirements. In order to carve a path for RTIC to be used in
such systems and industies, the tooling support needs to improve drastically.

Also the lack of easy-to-use tools could hinder the community to test their
applications. Most tools for schedulability analysis are framework agnostic but
because of that it requires a complete model of the system which are usually
described in less known query languages. But a schedulability tool made
specifially for RTIC with both practicallity and easy-of-use in mind could lead
to more community-driven projects to be tested.

% something community, make it easy to test

\section{Problem Definition}
This thesis explores the possibility of developing a tool to automatically
calculate the worst-case response times (WCRT) using the symbolic execution
engine KLEE, for tasks in an RTIC application in order to check if the system
is schedulable or not. As well as explore what the limitations of such an
approach would be. The aim is to showcase the work practically on actual
working applications.

\section{Delimitiations}
There will be some limitation to this tool to fit the time frame of the course.
A minimal subset of the features of RTIC will be supported at a start, to try
to prove that a tool like this can work. If there is time left, further
features should be looked into.

\section{Thesis structure}
