In computing a \emph{task} or sometimes process, is a unit of work that is
executed or will be executed by the processor in a computer or microcontroller.
For many real-time computing systems, the tasks running on them are liable to
meet specific deadlines. I.e.\ the tasks are bound to finish their execution or
respond within a certain time frame. This is especially true for smaller
embedded systems. This might be due to hardware constraints; the system will
not be able to work correctly if the system is over-utilized. Or it could be
critical that certain tasks generate a response in time, e.g.\ a sensor reading
or an actuator. In such scenarios, a task that fails to meet its deadline could
lead to system failures.

To make sure tasks adheres to their deadlines, many real-time systems utilizes
a \emph{scheduler} that schedules all tasks. \emph{Scheduling} is the act of
assigning system resources, such as processor time or hardware peripheral
access to the tasks in an orderly fashion. Many algorithms and policies to
schedule tasks currently exists and have been mathematically proven to be
correct.  E.g.\ rate-monotonic scheduling\cite{ratemonotonic}, earliest
deadline first scheduling\cite{edf} and stack resource policy\cite{srp} amongst
others.  The act of verifying that the system's tasks adhere to the scheduler
policy and are meeting their deadlines is called \emph{schedulability
analysis}.

One way to do this analysis is to calculate the longest time it takes for each
task to respond, i.e.\ the longest time from when the task is ready to execute
until the end of its execution and compare it to their respective deadlines.
Known as \emph{response time analysis}. In order to calculate the worst-case
response time (WCRT) the worst-case execution time (WCET) for the processes
needs to be known. The difference between WCRT and WCET is that WCET considers
the longest execution time of a task without any interruptions whereas WCRT
also considers interruptions by tasks which have higher priority to
run\cite{hardrealtimecomputingsystems}. If the response time is calculated to
be larger than the deadline then it is said to be \emph{unschedulable}.

In this thesis, the tool \emph{RAUK} for automatically verifying an embedded
application's schedulability using a measurement based WCET approach to
caclulate the WCRT of all tasks is presented. It currently works on
Run-Time Interrupt driven Concurrency (RTIC) applications, which is a framework
for programming real-time systems for embedded devices in the Rust programming
language.

\section{Background}
% mention current wcet approaches
Current approaches for automatic WCET calculation on real-time systems usually
utilizes two different methods.
\begin{enumerate}
    \item Measurement-based approach analyzing smaller code sections on hardware
    \item Static analysis of the program using a model of the system
\end{enumerate}
This is frequently referred as the hybrid approach and is used in safety critical
applications for the aviation industry\cite{rapita} amongst others.

However when it comes to schedulability analysis tools there are very few active
projects. Most such tools needs the user to manually model the system in order
to verify it like Cheddar\cite{cheddar}.

% mention rust/rtic approaches

% mention symbolic execution

%In this thesis the tool \emph{RAUK} for automatically verifying an embedded
%application's schedulability using a measurement based WCET analysis is presented.
%It currently only works on Run-Time Interrupt driven Concurrency (RTIC) applications,
%which is a framework for programming real-time systems.
%
%RAUK works in three steps. First it generates a test harness from the application
%for which the symbolic execution tool \emph{KLEE} can generate test vectors for.
%Then it generates a replay harness from the application which is used to replay
%the generated test vectors on actual hardware. Finally the WCET analysis is
%measured using those test vectors running on the hardware itself.


\section{Motivation}
The main objective is to make a streamlined prototype tool that can be used to
analyze and verify RTIC applications. The RTIC framework abstracts a lot of
details from the user. When the program is compiled it generates new code from
the source code of the user. As KLEE works with symbolic values to generate its
test cases, the input of the tasks in the generated code needs to be made
symbolic. This step should result in new code generation with symbolic values
which will be referred to as the test harness. KLEE can then generate test
cases for each task from the test harness. Finally each test case can be
replayed on the intended hardware or on emulated hardware. When replaying, the
response time can measured and then be compared to a theoretical value.
Currently there are no such tools available for this framework.

The tool should be able to do the following:
%
\begin{enumerate}
    \item Generate code from RTIC that can be run symbolically (test harness)
    \item Generate KLEE tests on the generated code
    \item Replay test cases on actual or virtual hardware to measure execution time
    \item Detailed schedulability analysis based on the measurements
\end{enumerate}

\section{Problem Definition}


\section{Delimitiations}
There will be some limitation to this tool to fit the time frame of the course.
A minimal subset of the features of RTIC will be supported at a start, to try
to prove that a tool like this can work. If there is time left, further
features should be looked into.

\section{Thesis structure}
